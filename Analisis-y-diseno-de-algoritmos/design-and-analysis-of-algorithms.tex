\documentclass{article}
\usepackage{fancyhdr}
\usepackage{amsmath, amssymb}
\usepackage{amsthm}

\author{Johan Madroñero Cuervo}
\title{Design \& Analysis of Algorithms}
\date{2023 - 1S}

\pagestyle{fancy}
\fancyhead{}
\fancyhead[HL]{\textbf{Design \& Analysis of Algorithms}}

\theoremstyle{definition}
\newtheorem{definition}{Definicion}

\begin{document}
\maketitle
\thispagestyle{empty}
\newpage
\tableofcontents
\thispagestyle{empty}
\newpage

\begin{flushleft}

\section{Lecture 1}
Tue Feb  7 08:14:06 2023

\subsection{Objetivos del curso}
\begin{itemize}

  \item[-] Encontrar el algoritmo más óptimo a los problemas propuestos.
  \item[-] Aumentar la oferta de asignaturas \textbf{disciplinar optativas}.
  \item[-] Preparar estudiantes para el circuito de competencias.
  \item[-] Mejorar las probabilidades de vinculación laboral.

\end{itemize}

\subsection{Pre-requisitos}

\begin{itemize}
  \item[-] Excelentes bases de programación.
  \item[-] Conocimientos de \textbf{Programación Orientada a Objetos}.
  \item[-] Cálculo y Estadística.
\end{itemize}

\subsection{Metodología}

\begin{itemize}

  \item[-] 4 talleres prácticos con 20\% cada uno.
  \item[-] Calificación de cada taller = $5.0 * \frac{realizados}{propuestos}$.
  \item[-] 1 Trabajo final con exposición oral de 20\%.

\end{itemize}

\subsection{Dedicación horaria}

144 horas totales de dedicación al curso.

\subsection{Bibliografía}

\begin{itemize}

  \item[-] Introduction to Algorithms Cormen, Thomas.
  \item[-] Algorithms Dasgupta.
  \item[-] Algorithm Desing Kleinberg.
  \item[-] Art of programming Contest Shamsul.
  \item[-] Mathematics for Computer Science Lehman, Eric.

\end{itemize}

\subsection{Consideraciones importantes}

\begin{itemize}

  \item[-] Casi nada del cuso es original.
  \item[-] Aprovechar grupo, monitor, asesorías.
  \item[-] No vamos a hacer demostraciones para todo algoritmo.

\end{itemize}

\subsection{Aseosrías}

\begin{itemize}

  \item[-] jmoreno1@unal.edu.co
  \item[-] daaristizabalg@uanl.edu.co
  \item[-] crhenaor@unal.edu.co
  \item[-] https://chat.whatsapp.com/Fio2A7fck8s75alHg2dgro

\end{itemize}

\subsection{Cursos y plataformas de interés}

\begin{itemize}

  \item[-] Codeforces y demás sitios de programación competitiva
  \item[-] Standford, MIT y de más sitios de cursos en línea.

\end{itemize}

\subsection{Conteo de asistentes}
\textit{``Siempre hay formas más eficientes para resolver problemas 
triviales".}
Se aborda la forma más eficiente de contar asistentes en un evento y finaliza
con una dinámica que cuenta los participantes.

\subsection{¿Qué es un algoritmo?}

\begin{definition}
\textbf{Algoritmo}
\textit{Conjunto ordenado de instrucciones bien definidas, no ambiguas y 
finitas con la finalidad de resolver un problema.}
\end{definition}

\subsection{Números de casa}
Se plantea un ejercicio que en las notas de Felipe era Street Numbers y se 
hace una explicación similar. 

\subsection{Medida de eficiencia}
Se utiliza la medida Big Oh. 

\subsection{Notación Big Oh}
En el curso se usa para eliminar discusiones de cómo se hacen las operaciones
en la máquina. En este caso nos fijamos en la forma de la función en tiempo 
empleado a medida que la entrada crece.

\subsubsection{Definición}

\begin{itemize}

  \item[-] Se considera el peor escenario.
  \item[-] Se realiza un análisis asintótico.
  \item[-] No se presta atención a términos constantes o de orden superior.

\end{itemize}

\subsection{Utilidad de la notación Big Oh}
ESta notación no es rigurosa y no es un predictor exacto. Este solo describe
la curva de las soluciones en tiempo con respecto a entradas.
Podemos observar que un algoritmo $O(N^2)$ será cuadrático.

\newpage

\section{Lecture 2}

Tue Feb 14 08:09:23 2023

\textit{"La primera solución que se nos ocurre o la solución más obvia 
generalmente no es la mejor"}
  - Julián Moreno.

\subsection{Cantidad de divisores}
Dado un número n, encontrar la cantidad de divisores de n diferentes de sí 
mismo.

Mejor caso: $O(\sqrt{N}$.

\subsection{Algoritmo de Euclides}
Encontrar el mínimo común múltiplo de dos enteros.

Usando el algoritmo de Euclides el mejor caso es: $O(\log(max(A,b))$.

\subsubsection{Criba de Eratóstenes}
Dado un número $n$, imprimir todos los números primos $p$  $| p < n$. El mejor 
caso es $O(N * \log(\log(N))$.

\subsection{Comparación de complejidades}
Es posible comparar algoritmos de forma teórica. 

\subsection{Problemas P y NP}

\begin{definition}

  \textbf{Problema tipo P:} Aquel que se puede solucionar y verificar con un
  algoritmo de orden polinómico. Como \textit{Números de calle}.

\end{definition}

\begin{definition}

  \textbf{Problema tipo NP:} Aquel que se puede verificar en tiempo polinomial,
  pero para solucionarlo requiere de un algoritmo de orden superior. 

\end{definition}

\begin{definition}

  \textbf{Problema tipo NP Completo: } Un problema C es \textbf{NP} si se 
  cumplen dos soluciones:

  \begin{enumerate}

    \item C es NP.
    \item Todo problema \textbf{NP} se puede reducir a C en tiempo polinomial. 
      Es decir que una solución para C (o parte de ella) se podría usar para 
      resolver los problemas \textbf{NP}.

  \end{enumerate}

\end{definition}

\subsubsection{SAT o B-SAT}
Consiste en dterminar si una fórmula proposicional se puede hacer verdadera 
mediante una determinada asignación de valores de verdad a sus variables. 


\end{flushleft}

\end{document}
