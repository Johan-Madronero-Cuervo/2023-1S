\documentclass{article}
\usepackage{fancyhdr}
\usepackage{amsmath, amssymb}
\usepackage{amsthm}

\author{Johan Madroñero Cuervo}
\title{Cátedra de Sistemas. Una visión histórico-cultural de la computación}
\date{2023 - 1S}

\pagestyle{fancy}
\fancyhead{}
\fancyhead[HL]{\textbf{Cátedra de Sistemas}}
\rfoot{\textbf{\thepage}}
\cfoot{}

\theoremstyle{definition}
\newtheorem{definition}{Definicion}

\begin{document}
\maketitle
\thispagestyle{empty}
\newpage
\tableofcontents
\thispagestyle{empty}
\newpage

\section{Lecture 1}

\subsection{Objetivos}

\begin{itemize}

  \item[-] Presentar una hisotria de la computación en una forma amena.
  \item[-] Materia en formato cátedra.
  \item[-] Minimizar pre-requisitos. Siendo así accesible para estudiantes
    nuevos, algo parecido está sucediendo con otras optativas en Sistemas.
  \item[-] Brindar un espacio para el debate de Historia, impacto de 
    tecnologías, equidad de género.
  \item[-] 

\end{itemize}

\subsection{Metodología}

\begin{itemize}

  \item[-] Películas.
  \item[-] Documentales.
  \item[-] Series.
  \item[-] Animes.
  \item[-] Libros. (2 en el semestre)

\end{itemize}

\subsection{Evaluación}

\begin{itemize}

  \item[-] Asistencia. \textbf{30\%}
  \item[-] Participación. \textbf{30\%}
  \item[-] Trabajo final. \textbf{40\%}

\end{itemize}

\subsection{Consideraciones importantes}

\begin{itemize}

  \item[-] Reflexionar en las secciones es la motivación principal.
  \item[-] Somos muchos, por favor mantener silencio durantes las 
    actividades.
  \item[-] Es la primera vez que se oferta la cátedra.

\subsection{Black Mirror: The Entire History of You. S1E3}
\begin{flushleft}
Es muy curioso que vuelva a ver este episodio, lo vi hace unos años
en clase de inglés. Presencié lo frágil que es la memoria. En mis 
recuerdos el protagonista asesinaba a Jonas y la última escena era 
él intentando remover el \textit{grain} antes que lo atrapara la 
policía.

Omitiendo detalles que en otro momento estaría encantado de discutir
como actuación, escenas, cast y demás. Opino que sería una tecnología
interesante, si es implementada correctamente haría una sociedad más 
justa y con el evidente dilema de vulnerar tu privacidad a favor de 
la seguirdad de las masas. Hubo un detalle en la película con la que 
no estoy de acuerdo, el bebé tenía un \textit{grain}. Dudo que sea 
obligatorio llevar uno porque había una chica que decidió vivir
\textit{grainless} después que le arrebataran el suyo. Por lo que 
asumiré que fue decisión de los padres y no estaría de acuerdo con 
el hecho de que el individuo no decida si quiere llevar uno.

Dudo que pueda ser implementada como está representada 
en la serie y habría un sin fin de problemas de seguridad por 
solucionar primero. Si intento hacer un paralelismo con la sociedad 
actual puedo centrarme en los teléfonos móviles. Muy poca gente no 
tiene uno y siento que el episodio de Black Mirror refleja los 
celos que se viven en la actualidad con otra tecnología.

\section{Lecture 2}
Fri Feb 17 14:04:58 2023

\subsection{}

\end{flushleft}


\end{itemize}

\end{document}
